\documentclass[sigconf, authorversion]{acmart}
%% These commands are for a PROCEEDINGS abstract or paper.
%\acmConference[Linz '20]{Linz '20: ACM WiSec}{July 08--10, 2020}{Linz, Austria}
%\acmBooktitle{Linz '20: ACM WiSec Conference
%  June 03--05, 2018, Woodstock, NY}
%\acmPrice{15.00}
%\acmISBN{978-1-4503-XXXX-X/18/06}
%%
%% Submission ID.
%% Use this when submitting an article to a sponsored event. You'll
%% receive a unique submission ID from the organizers
%% of the event, and this ID should be used as the parameter to this command.
%%\acmSubmissionID{123-A56-BU3}
%%
%% end of the preamble, start of the body of the document source.
\raggedbottom
\usepackage{subcaption}
\usepackage{graphicx}
\def\yanjun#1{{}{\color{blue} #1}}
\setcopyright{rightsretained}
\begin{document}
%%
%% The "title" command has an optional parameter,
%% allowing the author to define a "short title" to be used in page headers.
\title{Demo: iJam with Channel Randomization}
%%
%% The "author" command and its associated commands are used to define
%% the authors and their affiliations.
%% Of note is the shared affiliation of the first two authors, and the
%% "authornote" and "authornotemark" commands
%% used to denote shared contribution to the research.
\author{Jordan L. Melcher, Yao Zheng}
%\email{melcherj@hawaii.edu}
\orcid{0002-3338-6400}
\author{Dylan Anthony}
\orcid{0000-0000-0000}
\author{Matthew Troglia}
\affiliation{%
%   \institution{University of Hawai'i at Mānoa}
%   \streetaddress{2540 Dole St}
%   \city{Honolulu}
%   \state{Hawaii}
%   \postcode{96822}
}
\author{Thomas Yang, Alvin Yang, Samson Aggelopoulos}
\affiliation{%
  \institution{University of Hawai'i at Mānoa}
  \streetaddress{2540 Dole St}
  \city{Honolulu}
  \state{Hawaii}
  \postcode{96822}
}
\author{Yanjun Pan}
\orcid{0000-0000-0000}
\author{Ming Li}
\orcid{0000-0000-0000}
\affiliation{%
  \institution{University of Arizona}
  \streetaddress{Tucson, AZ 85721}
  \city{Tucson}
  \state{Arizona}
  \postcode{85721}
}


%%
%% The abstract is a short summary of the work to be presented in the
%% article.
\begin{abstract}
\input{Abstract.tex} 
\end{abstract}

%%
%% The code below is generated by the tool at http://dl.acm.org/ccs.cfm.
%% Please copy and paste the code instead of the example below.
%%
\begin{CCSXML}
<ccs2012>
<concept>
<concept_id>10002978.10002979.10002980</concept_id>
<concept_desc>Security and privacy~Key management</concept_desc>
<concept_significance>500</concept_significance>
</concept>
<concept>
<concept_id>10002978.10003014.10003017</concept_id>
<concept_desc>Security and privacy~Mobile and wireless security</concept_desc>
<concept_significance>500</concept_significance>
</concept>
</ccs2012>
\end{CCSXML}

\ccsdesc[500]{Security and privacy~Key management}
\ccsdesc[500]{Security and privacy~Mobile and wireless security}

\copyrightyear{2020}
\acmYear{2020}
\acmConference[WiSec '20]{13th ACM Conference on Security and Privacy in Wireless and Mobile Networks}{July 8--10, 2020}{Linz (Virtual Event), Austria}
\acmBooktitle{13th ACM Conference on Security and Privacy in Wireless and Mobile Networks (WiSec '20), July 8--10, 2020, Linz (Virtual Event), Austria}\acmDOI{10.1145/3395351.3401705}
\acmISBN{978-1-4503-8006-5/20/07}

%% A "teaser" image appears between the author and affiliation
%% information and the body of the document, and typically spans the
%% page.


%%
%% This command processes the author and affiliation and title
%% information and builds the first part of the formatted document.
\maketitle
\section{Introduction}
Physical layer key generation schemes aim to establish a shared key between two parties through an open channel eavesdropped by an adversary. The majority of the schemes leverage the changing wireless channel to generate the key bits, with a generation rate proportional to the entropy of the channel. The more random the channel the faster the key is generated. A few other designs seek to make key generation rate independent from the channel variation, by injecting artificial noise into the wireless channel to aid security \cite{Goel_S,Zhou_X,Liao_W,Li_Q,Goeckel_D}. The combination of the transmission and jamming signal introduces uncertainty to an eavesdropper in the form of noise \cite{Kim_Y,Vilela_JP,Hu_J}, and prevents the eavesdropper from obtaining the underlying key bits.

Gollakota et al. developed iJam, a robust friendly-jamming system which improved the physical-layer key generation for stationary wireless networks \cite{Gollakota_S}. The scheme lets the transmitter (Alice) interleave two identical sequences of bits while the intended receiver (Bob) jams one at random. Since Bob knows which bits are jammed and which ones are clear, he can select the clear bits and reconstruct the key, whereas the eavesdropper (Eve), being unaware of Bob's jamming targets, cannot recreate the key. To increase the randomness of the channel, the key is repetitively transmitted until Bob reconstructs the original message.

Although iJam can achieve secure key exchange in a static channel, Steinmetzer et al. identified a vulnerability by implementing a multi-antenna adversarial model designed to take advantage of the spatial variance to discern between the jammed and clean signals \cite{Steinmetzer_D}. The root of this vulnerability is due to the fact that the channel states between Alice and Eve remain unchanged in spite of Bob's jamming signal, which allows Eve, equipped with multiple antennas, to exploit the pilot or known symbols in Alice's transmission to estimate the channel and cancel its effect. Once Eve equalizes the channel, she may evaluate the signal divergence among multiple antennas to identify the clear symbol. Specifically, a symbol with a large divergence among multiple antennas is likely to be jammed and thus ignored. After a few iterations, clean transmission can be spliced together, and the key can be known.

We propose a defense mechanism against such an attack by combining channel randomization with prediction-based channel equalization. Channel randomization has been used to strengthen physical-layer security schemes, such as orthogonal blinding, that are known to be vulnerable against multi-antenna eavesdropper \cite{Hou_Y,Aono_T,Pan_Y_1,Hassanieh_H}. The method leverages a reconfigurable or moving antenna to create artificial changes in a wireless channel, resulting in unstable channel state information (CSI) between the transmitter and receivers. The prediction-based channel equalization cancels the randomizing effect for Bob by implementing an angle-of-departure (AoD) estimation algorithm to predict the CSI for any given antenna configuration. The combined results are that the channel state appears stable for Bob but continuously changing for Eve. Furthermore, the channel prediction eliminates the need for pilot based channel measurements, which denies Eve the opportunity to measure and cancel the changing effects.

\begin{figure*}[!htb]
\centerline{\includegraphics[width=\linewidth]{Images/iJam_system_model.png}}
\vspace{-15pt}
\caption{System Diagram}
\label{fig1}
\end{figure*}

%[contributions] When combining channel randomization with friendly jamming, the increase in entropy contributes to both the key generation rate and the overall security. Not only will a multi-antenna adversary have a smaller window to compare the variance, but it also will not be able to make the comparison in the first place. For this effort, 

In this demonstration, we developed and implemented the channel randomizing iJam system with a custom reconfigurable antenna and real-time AoD based channel prediction algorithm, to enhance the security of the key generation protocol. The audience is granted access to our remotely accessible platform via live-stream to control and observe the real-time effects of channel randomization on Bob and Eve. The increase in entropy contributes to both the key generation rate and the overall security. Our results indicate the CSI of the intended receiver bit error rate (BER) does not fluctuate when exposed to channel randomization while simultaneously worsening a multi-antennae adversary's BER to the level of random guessing.

% When combined with friendly jamming, the increase in entropy contributes to both the key generation rate and the overall security. Not only will a multi-antenna adversary have a smaller window to compare the variance, but it also will not be able to make the comparison in the first place. For this effort, we developed and installed a channel randomizing reconfigurable antenna in our remotely accessible cybersecurity lab to increase the robustness of the iJam key generation protocol. Our results indicate the CSI of the intended receiver bit error rate (BER) does not fluctuate when exposed to channel randomization while simultaneously minimizing a multi-antennae adversary to a BER of 50\%, equal to random guessing. In this demonstration, the audience will be granted access to our Cybersecurity lab via live-stream to see the real-time effects of channel randomization on the intended receiver and multi-antenna eavesdropper. 
\section{System Overview}
Consider an OFDM system shown in figure \ref{fig1}, where there are three active parties; Alice, who wants to establish a secret key with an intended receiver, Bob, and a passive eavesdropper, Eve. Alice, in green, equips a rotating antenna to randomize the channel, and a digital RF chain consisting of a compressed sensing-based AoD estimation algorithm and a precoding filter to predict and cancel the channel randomization effect for the intended receiver. Bob, in blue, comprises of two stationary antennas, one for receiving, and the other for jamming; while Eve, in red, uses two antennas and an adaptive filter to exploit the spatial variance of the jammed signal. 
%Consider an OFDM system shown in Fig. \ref{fig3}, where there are three active parties; Alice, who wants to establish a secret key with an intended receiver, Bob, and a passive eavesdropper, Eve. Bob is equipped with regular omnidirectional antenna(s) (OAs), while the eavesdropper Eve can have any types of antennas, including OAs, reconfigurable
%antenna(s) (RAs) and etc.. To achieve channel randomization, the transmitter
%Alice is equipped with reconfigurable antenna(s). Similar to existing works \cite{Pan_Y_1} \yanjun{(add more citations)}, the channels between Alice and Bob, and Alice and Eve are assumed to remain static.

%Two physical layer security methods are adopted, a friendly jamming scheme called iJam, and channel randomization. The iJam protocol works by randomly selecting several samples in the transmission to jam then repeating the transmission while jamming a different set of samples until all clean samples are received and reconstructed. With the implementation of OFDM, the additive white Gaussian noise disguises part of the intended signal. This process alone successfully demonstrates that it prevents the single-antenna adversary from obtaining any information about a key exchange \cite{Gollakota_S}. However, in \cite{Steinmetzer_D}, a vulnerability exposes that an adversary with multiple antennas can accurately determine which samples are jam eavesdroppers by observing the environment's spatial diversity. The algorithm takes liberty in assuming that the channel between Alice and Bob, and Alice and Eve remains static. Channel randomization augments iJam by changing the channel state before the necessary coherent time needed by Eve to determine to separate the clean and noisy time samples \cite{Pan_Y_1}. An angle of departure (AoD) algorithm estimates and predicts the channel. In this demonstration, Alice is equipped with a single rotating reconfigurable logarithmic antenna, as seen in Figure 2; Bob has two static logarithmic antennae, one for receiving, the other for jamming; and Eve has two static antennas. Figure 1 Displays the system overview.
 
%\section{Review of iJam}
%iJam is a novel static channel physical layer key generation method proposed by Gollakota et al. \cite{Gollakota_S}. In this scheme, Alice sends a clean OFDM transmission to Bob. And Bob simultaneously transmits the OFDM generated jamming signal at randomly selected time intervals. Alice repeats the same message until Bob has successfully jammed all of the time intervals. This process alone successfully prevents the single-antenna adversary from obtaining any information about a key exchange \cite{Gollakota_S}. However, in \cite{Steinmetzer_D}, a vulnerability exposes that a multi-antenna eavesdropper Eve can compare the captured waveforms from both of her antennas to differentiate between clean and jammed samples. In this way, the clean transmission can be spliced together after few transmissions, and the key can be known to Eve.


% \section{Implementation}
% \begin{figure*}[t]
% \centering
% \begin{minipage}{.3\textwidth}
%   \centering
%   \includegraphics[width=\linewidth]{Images/Physical setup.jpg}
%   \captionof{figure}{Network Diagram}
%   \label{fig:test1}
% \end{minipage}%
% \hspace{0.07\textwidth}
% \begin{minipage}{.3\textwidth}
%   \centering
%   \includegraphics[width=\linewidth]{Images/Network Diagram.PNG}
%   \captionof{figure}{Cybersecurity Lab}
%   \label{fig:test2}
% \end{minipage}
% \hspace{0.07\textwidth}
% \begin{minipage}{.3\textwidth}
%   \centering
%   \includegraphics[width=\linewidth]{Images/Network Diagram.PNG}
%   \captionof{figure}{Cybersecurity Lab}
%   \label{fig:test2}
% \end{minipage}
% \end{figure*}

\subsection{Channel Randomization}
The channel randomization is implemented by rotating a log-periodic dipole array (LPDA) antenna using a stepper motor. The half-width beam angle of the LDPA antenna's main lobe is approximately 60 degrees. Hence, rotating the antenna by 60 degrees can significantly change the channel state. The rotating speed of the step motor can vary from 1 RPM to 5 RPM. The rotation is carried out in sync with the data transmission, which prevents Eve from obtaining a channel equalization filter to correct the randomization effects.

\subsection{Channel Prediction and Equalization}
In our scheme, Alice predicts and cancels the channel randomization effect for Bob with a compressed sensing-based AoD estimation algorithm. The CSI between Alice and Bob (or Eve) is due to the combined result of multipath components and antenna patterns. The compressed sensing algorithm allows Alice to estimate the 360-degree AoD vector, which defines the multipath components. Given that Alice knows the antenna pattern for a specific antenna mode, the CSI can be computed as the inner product of the AoD distribution and the antenna pattern vector.
%The scatters in the environment create multiple channel components over different AoDs. 

To predict the wireless channel, Alice selects every antenna mode then transmits pilot symbols to Bob at each mode. Bob measures the corresponding CSI and sends it back to Alice using implicit feedback. Alice then estimates the AoD vector with the compressed sensing algorithm. Once known, Alice can predict the CSIs for all unused antenna modes and computes the corresponding precoding filter to cancel the changing channel effects for Bob.

% For this experiment, Alice must initialize the angle of departure (AoD) algorithm to provide flawless communication with Bob. Under most circumstances, the AoD is stable across the entire coherence time. By randomizing the channel, Alice first needs to capture the physical channel and predict the maximum effect of all unique channel modes of the reconfigurable antenna. Alice selects multiple antenna modes and sends them to Bob. Bob then measures all of the related channel state information (CSI) and sends it back to Alice using implicit feedback. With this information, Alice can estimate all channels for all future antenna modes. For the entire coherence time, Alice can cancel the changing effects for the intended receiver.

\subsection{Key Generation}
With the addition of channel randomization and prediction we created a less complicated physical layer key generation method. In the original scheme, iJam, Alice and Bob switched roles between transmitting and jamming to prevent adversaries who were unaffected by the jamming signals. Channel randomization replaces this method. In our scheme, Alice sends clean OFDM symbols to Bob. Simultaneously, Bob transmits jamming signal at randomly selected time intervals. Alice repeats the same message until Bob has successfully jammed every bit. During this process, Eve compares the captured waveforms from both of her antennas to differentiate between clean and jammed samples.

The bandwidth and frequency are carefully selected to optimize the trade-off between the speed and effect of randomization. The rotating antenna has a limit of 5 RPM with five unique channel modes. To put less strain on the AoD algorithm while providing ample security, 16-QAM is selected. To successfully transmit the repeated key before a new channel mode is selected, the transmission bandwidth is limited to 3.4 KHz. The key bits are randomly generated at Alice's side and shared with Bob through the software detailed in the appendix.

%a 3.4Khz bandwidth if 16QAM is chosen fixed 5 RPM
% the larger the QAM the more precise the equalization has to be

%The execution of this experiment was performed with a program development and execution system called Labview. The graphical programming environment was necessary to quickly and easily program every facet of the experiment which include, equipment control, OFDM transmission, RF generation, program physical layer security algorithms, and real-time collection, observation and analysis of data thus giving birth to a program which implements a channel randomization protocol with the mission of augmenting iJam, a physical wireless security key generation protocol to prevent a multi-antenna eavesdropping attack.
% \subsection{iJam}
% iJam is a novel static channel physical layer key generation method. Firstly, Alice sends a clean OFDM transmission to Bob. Simultaneously, Bob transmits the OFDM generated jamming signal at randomly selected time intervals. Alice repeats the same message until Bob has successfully jammed all of the time intervals. During this process, Eve compares the captured waveforms from both of her antennae to differentiate between clean and jammed samples. However, with the implementation of channel randomization, Eve cannot make the distinction. 
%\subsection{Channel Randomization}
%Channel randomization has proven effective in augmenting other physical wireless security methods \cite{Pan_Y_2,Pan_Y_1}. For this experiment, Alice must initialize the Angle of Departure (AoD) algorithm. Under most circumstances, the AoD is stable across the entire coherent time. However, by randomizing the channel, Alice first needs to capture the physical channel, predict the maximum effect of all unique channel modes of the reconfigurable antenna within a coherent time. Alice selects multiple antenna modes and sends them to Bob. Bob then measures all of the related CSI and sends it back to Alice using implicit feedback. With this information, Alice can estimate all channels for all future antenna modes. For the entire coherent time, Alice can cancel the changing effects for the intended receiver. The continually changing channel state prevents Eve from initiating its spatial diversity algorithm. Without a stable channel, Eve cannot accurately compare the transmissions at the differing locations and can only rely on random guessing.

% \begin{figure*}[h]
%   \includegraphics[width=0.25\linewidth]{Images/LabVeiw1_fix.png}
%   \hfill
%   \includegraphics[width=0.25\linewidth]{Images/labview2.png}
%   \hfill
%   \includegraphics[width=0.25\linewidth]{Images/Figure3_sq.jpg}
%   \caption{LabView GUI}
%   \Description{caption}
% \end{figure*}

\begin{figure*}[t]
%\centering
\begin{subfigure}[t]{0.23\linewidth}
  \centering
  \includegraphics[width=\linewidth]{Images/network_diagram.png}
  \caption{}
\end{subfigure}
\hspace{0.05\textwidth}
\begin{subfigure}[t]{0.23\linewidth}
  \centering
  \includegraphics[width=\linewidth]{Images/physical_setup.jpg}
  \caption{}
\end{subfigure}%
\hspace{0.05\textwidth}
\begin{subfigure}[t]{0.23\linewidth}
  \centering
  \includegraphics[width=\linewidth]{Images/alice_pattern.png}
  \caption{}
\end{subfigure}
\begin{subfigure}[t]{0.23\linewidth}
  \centering
  \includegraphics[width=\linewidth]{Images/LabVeiw1_fix.png}
  \caption{}
\end{subfigure}
\hspace{0.05\textwidth}
\begin{subfigure}[t]{0.23\linewidth}
  \centering
  \includegraphics[width=\linewidth]{Images/labview2.png}
  \caption{}
\end{subfigure}%
\hspace{0.05\textwidth}
\begin{subfigure}[t]{0.23\linewidth}
  \centering
  \includegraphics[width=\linewidth]{Images/Figure3_sq.jpg}
  \caption{}
\end{subfigure}
\caption{From top to bottom, left to right: (a) Network setup for remote demonstration. (b) Physical setup with Alice on the right, Bob and Eve on the left. (c) The radiation pattern of the LDPA antenna. (d-f) LabView panels to visualize CSIs and data transmissions.}
\label{fig2}
\end{figure*}
%%
%% The acknowledgments section is defined using the "acks" environment
%% (and NOT an unnumbered section). This ensures the proper
%% identification of the section in the article metadata, and the
%% consistent spelling of the heading.
\begin{acks}
% Dr. Zheng's work was partly supported by NSF grant NSF CNS-1948568.
% Dr. Li's work was partly supported by ARO grant W911NF-19-1-0050.
This work is partly supported by NSF grants CNS-1948568, DGE-1662487, ARO grant W911NF-19-1-0050, and Naval Information Warfare Center Pacific.
\end{acks}
%%
%% The next two lines define the bibliography style to be used, and
%% the bibliography file.
\bibliographystyle{ACM-Reference-Format}
\bibliography{reference}
\appendix
\section{Demonstration Description}
The demonstration platform, consisting of four software-defined radio(s), one-directional LDPA antenna installed on a rotary actuator, three omnidirectional antennas, and one computer server running a customized LabView VI, is hosted in the Cybersecurity Laboratory at the University of Hawai'i at Mānoa (UHM) as displayed in figure \ref{fig2}(b). As illustrated in figure \ref{fig2}(a), remote access to the platform is available through a Virtual Private Network (VPN) gateway implemented on an Amazon EC2 instance. Once connected to the platform, a user can launch the Labview program to rotate the antenna, initiate communications between the USRPs, visualize the wireless signal, and collect the data-traces for offline analysis. A live broadcast will verify the antenna position.

The Labview graphical user interface (GUI), made available to the public, contains the controls and displays necessary for the demonstration. The VI, shown in figure \ref{fig2}(d-f), allows users to perform experiments and analysis. Users can control IQ rate, transmission frequency and gain, bit number generation, rotation rate, and experiment duration. The graphs in figure \ref{fig2}(d), depict the transmitter signal constellation, symbol magnitude, and phase. Figure \ref{fig2}(e) displays magnitude and phase plots for signal reception verification of one of the three receivers. Most importantly, in figure \ref{fig2}(f), Bob and one of Eve's antennas compare their CSI. Two scenarios will run with and without channel randomization to observe the drastic effect on an eavesdropping attack.
%The demonstration platform, consisting of four software-defined radio(s), one directional antenna installed on a rotary actuator, three omnidirectional antennas, and one computer server running a customized LabView VI, is hosted in the Cybersecurity Laboratory at the University of Hawai'i at Mānoa (UHM). The platform can be remotely accessed through a Virtual Private Network (VPN) gateway implemented on an Amazon EC2 instance. Once connected to the platform, A user can launch the Labview VI to rotate the antenna, initiate communications between the USRPs, visualize the wireless signal, and collect the data-traces for offline analysis. The user can further launch a live-view to verify the position of the antennas. The Labview VI, shown in Figure 4, contains the controls and displays necessary for the demonstration. 

%The antenna and rotator controls appear on the left half of the image, and include, selecting the IP address for each USRP, IQ rate, frequency, gain, and bit number. Rotator controls include rotation rate, time of rotation, initializing to the current position, and resetting the antenna to the initialized position and starting the rotation. The demonstrator has the option to record a session for a specific amount of time. The real-time signal constellation, magnitude, phase, unequalized, and equalized data are displayed to verify that each antenna is performing nominally. To confirm that channel randomization protects the iJam key generation protocol, the continuous CSI data at Bob and both Eve antennae. During the demonstration attention is drawn to the CSI plots. The Eve phase plot is both constantly changing and differ from each other compared to that of Bob’s CSI phase plot, which remains static.


\end{document}

%%
%% End of file `sample-authordraft.tex'.
 