\section{Demonstration Description}
The demonstration platform, consisting of four software-defined radio(s), one-directional LDPA antenna installed on a rotary actuator, three omnidirectional antennas, and one computer server running a customized LabView VI, is hosted in the Cybersecurity Laboratory at the University of Hawai'i at Mānoa (UHM) as displayed in figure \ref{fig2}(b). As illustrated in figure \ref{fig2}(a), remote access to the platform is available through a Virtual Private Network (VPN) gateway implemented on an Amazon EC2 instance. Once connected to the platform, a user can launch the Labview program to rotate the antenna, initiate communications between the USRPs, visualize the wireless signal, and collect the data-traces for offline analysis. A live broadcast will verify the antenna position.

The Labview graphical user interface (GUI), made available to the public, contains the controls and displays necessary for the demonstration. The VI, shown in figure \ref{fig2}(d-f), allows users to perform experiments and analysis. Users can control IQ rate, transmission frequency and gain, bit number generation, rotation rate, and experiment duration. The graphs in figure \ref{fig2}(d), depict the transmitter signal constellation, symbol magnitude, and phase. Figure \ref{fig2}(e) displays magnitude and phase plots for signal reception verification of one of the three receivers. Most importantly, in figure \ref{fig2}(f), Bob and one of Eve's antennas compare their CSI. Two scenarios will run with and without channel randomization to observe the drastic effect on an eavesdropping attack.
%The demonstration platform, consisting of four software-defined radio(s), one directional antenna installed on a rotary actuator, three omnidirectional antennas, and one computer server running a customized LabView VI, is hosted in the Cybersecurity Laboratory at the University of Hawai'i at Mānoa (UHM). The platform can be remotely accessed through a Virtual Private Network (VPN) gateway implemented on an Amazon EC2 instance. Once connected to the platform, A user can launch the Labview VI to rotate the antenna, initiate communications between the USRPs, visualize the wireless signal, and collect the data-traces for offline analysis. The user can further launch a live-view to verify the position of the antennas. The Labview VI, shown in Figure 4, contains the controls and displays necessary for the demonstration. 

%The antenna and rotator controls appear on the left half of the image, and include, selecting the IP address for each USRP, IQ rate, frequency, gain, and bit number. Rotator controls include rotation rate, time of rotation, initializing to the current position, and resetting the antenna to the initialized position and starting the rotation. The demonstrator has the option to record a session for a specific amount of time. The real-time signal constellation, magnitude, phase, unequalized, and equalized data are displayed to verify that each antenna is performing nominally. To confirm that channel randomization protects the iJam key generation protocol, the continuous CSI data at Bob and both Eve antennae. During the demonstration attention is drawn to the CSI plots. The Eve phase plot is both constantly changing and differ from each other compared to that of Bob’s CSI phase plot, which remains static.

