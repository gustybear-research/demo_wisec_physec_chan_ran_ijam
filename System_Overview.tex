\section{System Overview}
Consider an OFDM system shown in figure \ref{fig1}, where there are three active parties; Alice, who wants to establish a secret key with an intended receiver, Bob, and a passive eavesdropper, Eve. Alice, in green, equips a rotating antenna to randomize the channel, and a digital RF chain consisting of a compressed sensing-based AoD estimation algorithm and a precoding filter to predict and cancel the channel randomization effect for the intended receiver. Bob, in blue, comprises of two stationary antennas, one for receiving, and the other for jamming; while Eve, in red, uses two antennas and an adaptive filter to exploit the spatial variance of the jammed signal. 
%Consider an OFDM system shown in Fig. \ref{fig3}, where there are three active parties; Alice, who wants to establish a secret key with an intended receiver, Bob, and a passive eavesdropper, Eve. Bob is equipped with regular omnidirectional antenna(s) (OAs), while the eavesdropper Eve can have any types of antennas, including OAs, reconfigurable
%antenna(s) (RAs) and etc.. To achieve channel randomization, the transmitter
%Alice is equipped with reconfigurable antenna(s). Similar to existing works \cite{Pan_Y_1} \yanjun{(add more citations)}, the channels between Alice and Bob, and Alice and Eve are assumed to remain static.

%Two physical layer security methods are adopted, a friendly jamming scheme called iJam, and channel randomization. The iJam protocol works by randomly selecting several samples in the transmission to jam then repeating the transmission while jamming a different set of samples until all clean samples are received and reconstructed. With the implementation of OFDM, the additive white Gaussian noise disguises part of the intended signal. This process alone successfully demonstrates that it prevents the single-antenna adversary from obtaining any information about a key exchange \cite{Gollakota_S}. However, in \cite{Steinmetzer_D}, a vulnerability exposes that an adversary with multiple antennas can accurately determine which samples are jam eavesdroppers by observing the environment's spatial diversity. The algorithm takes liberty in assuming that the channel between Alice and Bob, and Alice and Eve remains static. Channel randomization augments iJam by changing the channel state before the necessary coherent time needed by Eve to determine to separate the clean and noisy time samples \cite{Pan_Y_1}. An angle of departure (AoD) algorithm estimates and predicts the channel. In this demonstration, Alice is equipped with a single rotating reconfigurable logarithmic antenna, as seen in Figure 2; Bob has two static logarithmic antennae, one for receiving, the other for jamming; and Eve has two static antennas. Figure 1 Displays the system overview.
