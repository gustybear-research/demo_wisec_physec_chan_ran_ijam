\section{Introduction}
Physical layer key generation schemes aim to establish a shared key between two parties through an open channel eavesdropped by an adversary. The majority of the schemes leverage the changing wireless channel to generate the key bits, with a generation rate proportional to the entropy of the channel. The more random the channel the faster the key is generated. A few other designs seek to make key generation rate independent from the channel variation, by injecting artificial noise into the wireless channel to aid security \cite{Goel_S,Zhou_X,Liao_W,Li_Q,Goeckel_D}. The combination of the transmission and jamming signal introduces uncertainty to an eavesdropper in the form of noise \cite{Kim_Y,Vilela_JP,Hu_J}, and prevents the eavesdropper from obtaining the underlying key bits.

Gollakota et al. developed iJam, a robust friendly-jamming system which improved the physical-layer key generation for stationary wireless networks \cite{Gollakota_S}. The scheme lets the transmitter (Alice) interleave two identical sequences of bits while the intended receiver (Bob) jams one at random. Since Bob knows which bits are jammed and which ones are clear, he can select the clear bits and reconstruct the key, whereas the eavesdropper (Eve), being unaware of Bob's jamming targets, cannot recreate the key. To increase the randomness of the channel, the key is repetitively transmitted until Bob reconstructs the original message.

Although iJam can achieve secure key exchange in a static channel, Steinmetzer et al. identified a vulnerability by implementing a multi-antenna adversarial model designed to take advantage of the spatial variance to discern between the jammed and clean signals \cite{Steinmetzer_D}. The root of this vulnerability is due to the fact that the channel states between Alice and Eve remain unchanged in spite of Bob's jamming signal, which allows Eve, equipped with multiple antennas, to exploit the pilot or known symbols in Alice's transmission to estimate the channel and cancel its effect. Once Eve equalizes the channel, she may evaluate the signal divergence among multiple antennas to identify the clear symbol. Specifically, a symbol with a large divergence among multiple antennas is likely to be jammed and thus ignored. After a few iterations, clean transmission can be spliced together, and the key can be known.

We propose a defense mechanism against such an attack by combining channel randomization with prediction-based channel equalization. Channel randomization has been used to strengthen physical-layer security schemes, such as orthogonal blinding, that are known to be vulnerable against multi-antenna eavesdropper \cite{Hou_Y,Aono_T,Pan_Y_1,Hassanieh_H}. The method leverages a reconfigurable or moving antenna to create artificial changes in a wireless channel, resulting in unstable channel state information (CSI) between the transmitter and receivers. The prediction-based channel equalization cancels the randomizing effect for Bob by implementing an angle-of-departure (AoD) estimation algorithm to predict the CSI for any given antenna configuration. The combined results are that the channel state appears stable for Bob but continuously changing for Eve. Furthermore, the channel prediction eliminates the need for pilot based channel measurements, which denies Eve the opportunity to measure and cancel the changing effects.

\begin{figure*}[!htb]
\centerline{\includegraphics[width=\linewidth]{Images/iJam_system_model.png}}
\vspace{-15pt}
\caption{System Diagram}
\label{fig1}
\end{figure*}

%[contributions] When combining channel randomization with friendly jamming, the increase in entropy contributes to both the key generation rate and the overall security. Not only will a multi-antenna adversary have a smaller window to compare the variance, but it also will not be able to make the comparison in the first place. For this effort, 

In this demonstration, we developed and implemented the channel randomizing iJam system with a custom reconfigurable antenna and real-time AoD based channel prediction algorithm, to enhance the security of the key generation protocol. The audience is granted access to our remotely accessible platform via live-stream to control and observe the real-time effects of channel randomization on Bob and Eve. The increase in entropy contributes to both the key generation rate and the overall security. Our results indicate the CSI of the intended receiver bit error rate (BER) does not fluctuate when exposed to channel randomization while simultaneously worsening a multi-antennae adversary's BER to the level of random guessing.

% When combined with friendly jamming, the increase in entropy contributes to both the key generation rate and the overall security. Not only will a multi-antenna adversary have a smaller window to compare the variance, but it also will not be able to make the comparison in the first place. For this effort, we developed and installed a channel randomizing reconfigurable antenna in our remotely accessible cybersecurity lab to increase the robustness of the iJam key generation protocol. Our results indicate the CSI of the intended receiver bit error rate (BER) does not fluctuate when exposed to channel randomization while simultaneously minimizing a multi-antennae adversary to a BER of 50\%, equal to random guessing. In this demonstration, the audience will be granted access to our Cybersecurity lab via live-stream to see the real-time effects of channel randomization on the intended receiver and multi-antenna eavesdropper. 